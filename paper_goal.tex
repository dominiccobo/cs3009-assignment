\section{Paper's Goal}
\subsection{Why was this research conducted?}

The abstract and introduction shows research was conducted to address the gap in literature between ``studying user's privacy concerns and designing privacy tools by experts''.

Supported by elaboration on related work, they identify that work within this area has strongly been conducted by experts and researchers with little to no focus on user's real needs. 
\subsection{Who are the potential beneficiaries?}

\begin{description}
	\item[IOT Manufacturers] Within the literature, it is shown that this research may be of use to manufacturers to show what users may be willing to tolerate and what they wish to have more control over.
	
	\item[Home Owners IOT Users] A large focus of the study was on this stakeholder. It is shown the desire for greater control and insight into the privacy of smart home devices. 
	
	\item[Policy Makers and Government Bodies] The authors heavily discuss the policy implications of their research and highlight its usefulness to policy makers in regulating the appropriate behaviour of device manufacturers. They also allude to how it may be used to contrast existing worldwide privacy laws and their suitability against the user-sourced factors this study has produced. 
	
	\item[Passive IOT Users] IOT privacy is just alike smoking. Smokers are not the only affected parties by their actions, passive bystanders can be subject to side effects as well. The paper discusses the concerns of some participants in that there is little to no consideration given for the data of those visiting.
	
	\item[Academics] The authors highlight how this paper may provide a foundation and direction for future research. They recommend focusing on the addressed limitations such as scale and continuing to explore and study the application of UCD to this field.
\end{description}


