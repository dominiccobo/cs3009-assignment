\section{The Problem}

\subsection{Background Knowledge}

The chosen paper by Yao et al. \parencite*{Yao:2019:DMC:3290605.3300428} aims to explore the non-researched area of understanding how users address their own privacy concerns. They focus on the importance of user-centred privacy design (UCD) for smart homes. 


Within the introduction the paper raises the understanding from (9, 28) that a large amount of data is collected by smart homes which enables the potential inference of sensitive information. Further elaboration is made revealing how users are concerned about the privacy of this (6, 40, 42, 18, 21).

The authors then proceed to elaborate on related work which they categorise into two.

\subsubsection{Concerns and Risks}

Previous research highlights several areas of concern and risk.
Data has been shown to be at risk of theft and/or re-appropriation for malicious or non-intended goals (3, 1, 2). Data is often used for statistical inference of sensitive user information (45, 6, 42). Ultimately the expectations of concern, risk and trust vary according to the original intention of usage for the data (40, 43).
	
\subsubsection{Mitigating Mechanisms}

The authors contrast the aforementioned concerns and risks with other research performed by experts and researchers. They show how there have been successful attempts at adding difficulty to inference through the introduction of noise (2, 11, 41, 39).

It is also shown how a reduction of improper access to this data can be achieved through 29, 3, 8.

Work showing how transparency of function can be achieved is shown by 10, 26.

Reduction of user-burden on keeping systems safe and secure is shown in 23.


All of the above literature draws plentifully from the fields of information and cyber security,  as well as Systems networking.

\subsection{What does this paper bring to the table?}

This paper shows that privacy ideals vary from user to user, with each taking different approaches to mitigate associated problems from their perspective. 

They further elaborate on these ideals through the use of User-Centred Design to introduce six key factors that facilitate privacy design for a smart home.
