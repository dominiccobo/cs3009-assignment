\section{The Approach}

\subsection{The Methodology}

The method section displays how the research method used was co-design. Participants were recruited by Word of Mouth, online advertisement boards, libraries and senior citizen centres to partake in an advertised ``Design study for Smart Home Tech'' in exchange for monetary compensation. 

A pre-screening was conducted where participants were grouped based on the authors measurement of experience with smart home devices; this was based on whether they had used, planned to use or never used a smart home device.

Two sessions were conducted. The first focused on understanding the smart home privacy perception in a group-based discussion. The second focused on co-designing and prototyping a solution to these perceptions through co-creation amongst others in the group of experientially-similar peers. 

Following the conclusion of these sessions the data was analysed. A transcription of the recorded audio from the sessions was conducted. This transcript was read by multiple people then codified in a collaborative joint session at sentence level. 

The code was then supplemented individually and compared for potential differences using the Kappa measurement. 

Each code entry was then codified into a theme with cross-examination to ensure there was an agreement as to each categorisation.

The image data from the produced sketches was codified using Poole's methodology. Co-authors coded elements in every design. 
Codes drawn from these sketches were then also added to the existing themes with the Kappa analysis applied to ensure consistency. 

\subsection{Methodology Suitability}

The authors noted that they omitted the word privacy from the advertisement to ensure there was no bias.

Using word of mouth as a recruitment could face likely introduce of bias as these may be people of close contact with the authors and thus potentially share the same views or same level of understanding. This bias however may be outweighed by that of including other sources.

- some research questions appear as potentially leading (however, authors consider this)

- omission of sub-18 age group

- lo-fi paper design likely to hide potential further implications and not consider feasibility 

- Small dataset

- No measurement of true experience levels (however, suggestion to educate to normalise results)


\subsection{Assumptions}

- Sample recruited is representative of population 

- Themes produced were truly representative of wider privacy and smart home motifs.

\subsection{Applied Rigour}

The paper constantly evaluates the suitability of the method applied. There is ample discussion into situations where bias may occur and the actions taken to mitigate the effect of these. For example, the authors expose a potential leading question in their initial session however show that the effects have been considered and mitigated through a balancing subsequent question. 

There is a discussion of the limitations of the sample used. No participants in the sub-18 category were recruited despite related work identifying significant interest in this area. The sample size was also significantly small at only 25. The authors reveal that there was no focus on the wider stakeholder group and the focus was solely on the end user. 

They also show that the focus on a lo-fi approach could have restrained and limited concerns for feasibility and other further implications.