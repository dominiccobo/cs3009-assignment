\section{The Approach}

\subsection{The Methodology}

The method section displays how the research methods used were several; These included co-design and interviewing. These were all conducted within a focus group. Participant recruitment was done through word of mouth, online advertisement boards, libraries and senior citizen centres to partake in an advertised ``Design study for Smart Home Tech'' in exchange for monetary compensation. 

A pre-screening was conducted where participants were grouped based on the authors measurement of experience with smart home devices; this classification was based on whether they had used, planned to use or never used a smart home device.

Two sessions were conducted. The first focused on understanding the smart home privacy perception in a group-based discussion. The second focused on co-designing and prototyping a solution to these perceptions through co-creation amongst others in the group of experientially-similar peers. 

Following the conclusion of these sessions the data was analysed. A transcription of the recorded audio from the sessions was conducted. This transcript was read by multiple people then codified in a collaborative joint session at sentence level. 

The produced code was then supplemented individually and compared for potential differences using the Kappa measurement. 

Each code entry was then codified into a theme through cross-examination to ensure there was an agreement as to each categorisation.

The image data from the produced sketches was codified using the methodology from \cite{Poole:2008:RIU:1409635.1409662}. Co-authors coded elements in every design. 
Codes drawn from these sketches were then also added to the existing themes with the Kappa analysis applied to ensure consistency. It is important to note that the drafting of thematic categories was not statistically corroborated, but instead was non-descriptively co-verified.

\subsection{Methodology Suitability}

The authors noted that they omitted the word privacy from the advertisement to ensure there was no bias.

Using word of mouth as a recruitment method could face likely introduction of bias as these may be people of close contact with the authors and thus potentially share the same views or same level of understanding. This bias however may be outweighed by that of including other sources.

Some questions asked within the session are noted as being potentially leading; however, the authors note that this is intentional and is further balanced with another question.

The selection of a sample appears to omit the potential to consider stakeholders other than a specific age group of users above 18 years of age. This is of interest, despite the potential ethical implications of surveying minors, as the background identifies these as a key party involved in the aforementioned for Smart Home Privacy issues. However, it's questionable how much attention those under 18 year would pay in a focus group and such another research methodology could be examined for this. In addition, the omission of the other stakeholders, seems to have allowed a greater focus on the study goal of understanding user-centric privacy concerns and mitigations.

The selection sample in itself appears to be rather small, and the authors do identify that further studies on a larger scale could aid this study. However, focus groups tend to focus on smaller samples, in exploration for a richer data extraction than a large group which may be the case of other research methods.

When separating the selection sample into experience levels, approaching this based on whether the participant is a current user, is aiming to use or has never used a smart home device is likely to yield inconsistent results. Not all those who are users hold equal levels of understanding of what they may be using; from the same angle non-users could be more experts than those who use them. The authors do however allude to a need for potential pre-survey education to normalise any issues around preconceptions.

The use of a focus group may enable group-based dynamics to expose more considerations that would have not been otherwise drafted in other approaches. The potential for a minority voice is also catered for after the initial designs by allowing each participant to work on their own design.

The choice of a lo-fi versus a high-fi approach, at the potential cost of hiding further implications and feasibility matters, could enable more expressive non-expert designs for those involved.

\subsection{Assumptions}

Ontologically, the authors approach their research with an understanding of reality being a social construct as opposed to an observable entity. This is illustrated though their intention to understand the many ways users may consider their privacy.

When understanding the epistemology of this paper, it can be seen that knowledge is derived from an understanding of the different approaches that may result as a part of human nature; this can be seen as interpretative. There is a desire to understand the meanings surrounding interactions people may have to protect their privacy. 

The study's result was the categorisation of design, interview transcriptions and notes into themes. The creation of these relies on a large assumption that these are truly representative of motifs within the broader Privacy and Smart Home Domains. 

When understanding how the authors consider smart home technology's role, there seems to be a skewing towards the notion of a social strategic choice. It shown through the approach to designs that participants are given the free will and can use technology in the means they see fit. This, alongside the omission of stakeholders such as manufacturers or and a lack of hi-fi design could hide the contextual problems such as lack of implementation and technology types. 

\subsection{Applied Rigour}
% Credibility

Whilst there is no a thorough satisfaction of every single criteria found with the varying approaches of qualifying rigour in qualitative research \parencite{lincolnGuba1985}, such as member checking, there is sufficient rigour to consider the paper credible.

Methodological triangulation is achieved through having a two stage method where data is collected within both the round-table interview and lo-fi prototype design stages. Investigator triangulation is achieved by authors corroborating, with the help of statistical tests such as Cohen's Kappa, that interpretation during codifying does not vary disproportionately. 

Unfortunately the authors do not seem to achieve full referential integrity as a result of a lack of attached raw archival of data, which would have enabled others to verify conclusions. Positively, the coding procedures and the transparency apply, do build on the credibility of the research. A thorough discussion of the codified results is also included which allows greater insight into the insights of how and what was coded.

% Transferability

Throughout the method and results section a thorough (thick) description was made of the context and behaviours of the study. For example, the authors define a what a smart home is as a starting context. where the interviews are conducted and typical interaction with a smart home. This demonstrates that the research is transferable.

% Dependability

Looking at the paper's dependability, a consistent application of the methods is appreciable. This is despite a little deviance from these where the authors have justified slight alterations to the approach in order to achieve a better sample including those with mobility impairment. This deviation appears to be well based on another paper's approach to the same problem.

% Confirmability

When examining the confirmability, it can be appreciated that there is a clear presence of a conformability audit, demonstrated through a clear map of methods allowing replication by any party.  

Interpretations and recommendations, showing the future potential of UCD applied to privacy concerns, are well supported by the results.  

Throughout the paper there a constant reflection; a thorough evaluation of suitability for the method applied. There is ample discussion into situations where bias may occur and the actions taken to mitigate the effect of these. For example, the authors expose a potential leading question in their initial session however show that the effects have been considered and mitigated through a balancing subsequent question. 

There is a discussion regarding limitations of the sample used. No participants in the sub-18 category were recruited despite related work identifying significant interest in this area. The sample size was also significantly small at only 25. The authors reveal that there was no focus on the wider stakeholder group and the focus was solely on the end user. 

They also show that the focus on a lo-fi approach could have restrained and limited concerns for feasibility and other further implications.

There is a suggestion that the produced codes were given a measured considered examination to be applied to the correct theme, but there is little understanding of how this was done.

\newpage